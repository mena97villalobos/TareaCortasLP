\documentclass[10pt,journal,compsoc]{IEEEtran}
\usepackage[spanish]{babel}
\usepackage[linguistics]{forest}
\usepackage{graphicx}
\usepackage{wrapfig}
\usepackage{listings}
\ifCLASSOPTIONcompsoc
  \usepackage[nocompress]{cite}
\else
  \usepackage{cite}
\fi

\ifCLASSINFOpdf
\else
\fi
\newcommand\MYhyperrefoptions{bookmarks=true,bookmarksnumbered=true,
pdfpagemode={UseOutlines},plainpages=false,pdfpagelabels=true,
colorlinks=true,linkcolor={black},citecolor={black},urlcolor={black},
pdftitle={Oz Mozart},
pdfsubject={Tarea Corta 4, Oz Mozart},
pdfauthor={Daniel, Wilbert, Anthony, Bryan},
}
\renewcommand{\lstlistingname}{Cuadro}
\lstset{
	extendedchars=true,
	frame = single, 
	language=Pascal, 
	framexleftmargin=3pt
}
\hyphenation{op-tical net-works semi-conduc-tor}

\begin{document}
\title{Oz Mozart}
\author{Daniel~Delgado,~\IEEEmembership{Estudiante,~ITCR,}
        Wilbert~Gonzales,~\IEEEmembership{Estudiante,~ITCR,}
        Anthony~Leandro,~\IEEEmembership{Estudiante,~ITCR,}
        and~Bryan~Mena,~\IEEEmembership{Estudiante,~ITCR}
}
\markboth{Lenguajes de Programaci\'on, Tarea Corta 4, Septiembre 2017}
{Shell \MakeLowercase{\textit{et al.}}: \LaTex}
\maketitle
\IEEEdisplaynontitleabstractindextext
\IEEEpeerreviewmaketitle

\section{Datos Historicos}
\begin{wrapfigure}{R}{0.3\textwidth}
	\centering
	\includegraphics[width=0.25\textwidth]{logo.png}
	\caption{\label{fig:Logo}Logo de Mozart}
\end{wrapfigure}
Concebido en 1991 por Gert Smolka en la universidad de Saarland y desarrollado en colaboraci\'on con Seif Haridi y Perter Van Roy en el SICS (Swedish Institute of Computer Science), desde 2005 recibe mantenimiento por \emph{Mozart Board}. Tiene herencia de:
\begin{itemize}
	\item Prolog
	\item Earlang
	\item LISP/Scheme
\end{itemize}
Oz es un lenguaje multiparadigma. Incluye las siguientes caracter\'isticas que lo hacen un lenguaje muy interesante para ense\~nar e investigar:\footnote{Datos tomados de \emph{Programming in Oz}, Ku\'snierczyk, W.}
\begin{itemize}
	\item Imperativo(Stateful) y funcional(Stateless)
	\item Data-Driven y Demand Driven programming
	\item Programaci\'on Relacional (L\'ogica) y Constraint-Propagation
	\item Concurrent and distributed programming
	\item Orientaci\'on a Objetos
\end{itemize}
\subsection{Mozart}
Mozart es un implementaci\'on de OZ, es un lenguaje de alto nivel desarrollado por la Universit\'e Catholique de Louvain para propositos educativos.

\section{Tipos de Datos\protect\footnote{Por comodidad, adjuntamos imagen con tipos de datos en el Ap\'endice A. Imagen tomada de \emph{Tutorial of Oz}}}
En Oz las variables no son variables, en OZ las variables son identificadores y no estan asignadas sino unificadas
\begin{itemize}
	\item Dynamically Type Language
	\item Cuando una variable es creada su tipo y su valor son desconocidos
	\item Solamente cuando una variables asociada con un valor se determina su tipo
\end{itemize}

\subsection{Estructuras B\'asicas de datos}
\begin{itemize}
	\item Numbers:
		\subitem Float: Es necesario que tengan decimales, en OZ 5.0 != 5
		\subitem Integer: OZ soporta formato binario, octal y hexadecimal para su representaci\'on
	\item Record: Compuesto de una etiqueta y un numero fijo de elementos
		\subitem Open Records: Igual que un Record pero con n\'umero variable de elementos 
		\subitem Agrupar datos
		\subitem Ejemplo de Record:
\begin{lstlisting}[language=Oz, caption = {Record}][linewidth=5.4cm]
<Etiqueta>(Feature:Field)
\end{lstlisting}
		Donde etiqueta es el nombre asociado al record, Feature es una etiqueta del elemento y field es el elemento, el conjunto de todas las etiquetas de Features se les llama arities (Algo similar a las keys en un diccionario de Python) 
		\subitem Se utiliza la notaci\'on "." para accesar a un elemento de un record. Ejemplo:
\begin{lstlisting}[language=Oz, caption = {Accesar al elemento de un Record}][linewidth=5.4cm]
{Browse <Etiqueta>.Feature}
\end{lstlisting}
		Donde etiqueta es el nombre del record y Feature la etiqueta del elemento que se quiere accesar
	\item Literals: Tipos donde sus miembros no tienen estructura interna
		\subitem *Atoms: Records vacios, solo que \'unicamente contine una etiqueta y no tiene ninguna caracter\'istica
		\subitem *Names: Es un identificador universal y \'unico, la unicamanera de crearlo es llamando a \verb|{NewName <Etiqueta>}|, su uso es importante ya que ayuda a la seguridad, como se ve en el arbol de tipos de datos un hijo de Name es Bool, esto hace que los valores true y false sean un Name por si solo, o sea \'unicos, universales e invariantes
	\item Tuplas
		\subitem Es un tipo de record, consiste de una etiqueta y valores
		\subitem Ejemplo
\begin{lstlisting}[language=Oz, caption = {Tupla}][linewidth=5.4cm]
<Etiqueta>(Feature:Field)
\end{lstlisting}
		\subitem En realidad las tuplas son records donde los features son numeros desde 1 hasta la cantidad de elementos de la tupla:
\begin{lstlisting}[language=Oz, caption = {Tupla}][linewidth=5.4cm]
<Etiqueta>(1:elemento1 2:elemento2)
\end{lstlisting}
	\item Listas: puede ser el atomo nil para representar una lista vacia o puede representar una tupla usando el operador infijo $\mid$ y dos argumentos, los cuales son la cabeza y la cola de la lista, otra representaci\'on utili es la lista cerrada representada por \verb|[ ]| y separando elementos por espacios. Tambien estan las listas representadas con los caracteres \verb|"| (doble comilla al inicio y al final) estos son los string
\end{itemize}
Algunas ideas sobre los tipos de datos anteriores:
\begin{itemize}
	\item \textbf{Chunks}
		\subitem Prmite al usuario introducir tipos de datos abstractos
	\item \textbf{Cell}
		\subitem Modificar el estado de la l\'ogica
	\item \textbf{Space}
		\subitem Resoluci\'on de problemas utilizando "\textit{Search Techniques}"
\end{itemize}

\section{Estructuras de Control}
\begin{figure}[h]
	\centering
	\includegraphics[width=0.5\textwidth]{struct.png}
	\caption{The Oz kernel language}
\end{figure}
\subsection{Operadores Condicionales}
\begin{tabular}{c p{3cm} p{5cm}}
	Operador & Significado\\
	\hline\hline\\
	$==$ & Igualdad\\
	$>$ & Mayor que\\
	$<$ & Menor que\\
	$>=$ & Mayor Igual que\\
	$<=$ &Menor Igual que\\
	\verb|\=| & Diferente\\
	\hline
\end{tabular}
\subsection{Declaraci\'on de variables}
Para la declaraci\'on de variables que pertenecen a un scope dado se utiliza
\begin{lstlisting}[language=Oz, caption = {Variables en un scope}][linewidth=5.4cm]
local X Y Z in S end
\end{lstlisting}
El c\'odigo anterior crear 3 variables (X, Y, Z) y ejecuta S. Usualmente las variables inician con may\'uscula seguido de cualquier cantidad de caracteres alfa numericos. Otra manera de declarar variavles es:
\begin{lstlisting}[language=Oz, caption = {Variables en un scope}][linewidth=5.4cm]
declare X Y Z in S
\end{lstlisting}
Lo que esto hace es que X, Y, Z sean visibles globalmente en S y en los estatutos que sigan a S
\par En Oz hay pocas maneras de asociar variables a un valor, la usual es utilizar el operador infijo $"="$, ahora bien, si una variable ya contiene un valor la operaci\'on es considerada un test.

\subsubsection{¿Qu\'e pasa si se hace X = Y?}
Cuando se crea una variable se le asigna un espacio en memoria, un nodo, este nodo al inicio tiene valor y tipo desconocido, cuando las referencias a esa variable no existen se inicia un proceso de garbage collection para liberar los nodos que utilizaba esta varaible, cuando se utiliza la operaci\'on $"="$ intentar\'a unificar los valores de X y Y copiando sus nodos. La operaci\'on $"="$ se conoce como \emph{incremental tell} o \emph{unification}, algunos de sus resultados dependiendo del contexto:
\begin{itemize}
	\item Si las etiquetas X y Y pertenecen al mismo nodo la operaci\'on esta completa
	\item Si X no esta asociado se unifica el nodo de X con el nodo de Y, o sea todas las referencias a X pasan a ser referencias a Y
	\item Si X y Y contienen Records Rx y Ry respectivamente:
		\subitem Si los records Rx y Ry tienen diferentes etiquetas o arities se lanza un exception
		\subitem De otra manera los features de los records son unificados
\end{itemize}

\subsection{Operador de Igualdad}
Para probar una igualdad se utiliza l c\'odigo:
\begin{lstlisting}[language=Oz, caption = {Variables en un scope}][linewidth=5.4cm]
{Value.'==' X Y R}
\end{lstlisting}
Lo que se hace es probar si X es igual a Y y dejar el resultado en R. La operaci\'on:
\begin{itemize}
	\item Retorna true si los elementos tienen la misma estructura y los mismos valores o si son referencias al mismo nodo en memoria
	\item Retorna false si los elementos tienen estructuras o valores diferentes
	\item Se suspende cuando los nodos son diferentes pero existe un elemento sin asociar a un valor. Como Oz es un lenguaje concurrente cuando pasa esto el hilo que ejecut\'o el c\'odigo se suspende tambien.
\end{itemize}
Tambien se puede utilizar el operador  "==" como infijo tal que R = X == Y donde se prueba si X y Y son iguales y se deja el resultado en R

\subsection{Estatutos IF}
\begin{lstlisting}[language=Oz, caption = {Variables en un scope}][linewidth=5.4cm]
if B then S1 else S2 end
\end{lstlisting}
Lo que hace:
\begin{itemize}
	\item Si B es true S1 se ejecuta
	\item Si B es false S2 se ejecuta
	\item Si B no es un valor Booleano una exception ocurre
	\item Si B no tiene un valor asociado el hilo ejecutando se suspende 
\end{itemize}
Importante mencionar que la palabra reservada \emph{skip} funciona como el \emph{continue} en Python, adem\'as existe la abreviaci\'on \emph{elseif} que vendr\'ia siendo algo similar al \emph{elif} en Python:
\begin{lstlisting}[language=Oz, caption = {Variables en un scope}][linewidth=5.4cm]
if B1 then S1 elseif B2 then S2 else skip end
\end{lstlisting}

\subsection{Estatutos CASE}
\begin{lstlisting}[language=Oz, caption = {Variables en un scope}][linewidth=5.4cm]
case E 
of Pattern_1 then S1 
[] Pattern_2 then S2 
[] Pattern_3 then S3
[] ...
...
else S end
\end{lstlisting}
\subsubsection{Semantica}
Lo que hace el estatuto case es evaluar E con los $patrones_i$ esto de izquierda-derecha y \emph{depth-first}. Lo que hace:
\begin{itemize}
	\item Si E hace match con el $patron_i$ y E no esta siendo utilizado la instrucci\'on $S_i$ se ejecuta
	\item Si E hace match con el $patron_i$ pero E se esta utilizando se suspende el hilo
	\item Si E no hace match con el $patron_i$ se intenta con el $patron_i+1$ asi hasta alcanzar el else que se ejecutaria por defecto
	\item Dado sea el caso que la parte del \emph{else} sea omitida si E no hace match con algun $patron_i$ se lanza un exception
\end{itemize}

\subsection{Procedimientos}
\begin{lstlisting}[language=Oz, caption = {Variables en un scope}][linewidth=5.4cm]
proc {P X1 ... Xn} S end
\end{lstlisting}
El c\'odigo anterior lo que hace es crear una lambda expression \'unica lo que lo hace diferente a todos los dem\'as procedimientos existentes, esa expresi\'on esta asociada con el valor P. Como dato curioso, la equicvalencia de procedimientos se realiza mediante su etiqueta o nombre. Ejemplo:
\begin{lstlisting}[language=Oz, caption = {Obtener el mayor entre dos n\'umeros}][linewidth=5.4cm]
local Max X Y Z in 
	proc {Max X Y Z}
		if X >= Y then Z = X 
		else Z = Y end 
	end 
	X = 5
	Y = 10
	{Max X Y Z} {Browse Z}
end
\end{lstlisting}

\section{Caracter\'isticas}
\begin{itemize}
	\item Compilado o Interpretado, implementado en la plataforma Mozart
	\item Seguro, las entidades son creadas y pasadas explicitamente, esto significa que una aplicaci\'on no puede accesar o dar acceso a referencias que no se le han dado o creado en si misma
	\item Multiparadigma
		\subitem Orientaci\'on a Objetos
		\subitem Programaci\'on L\'ogica
		\subitem Programaci\'on Concurrente
			\subsubitem Threads Din\'amicos
\end{itemize}

\section{Ventajas}
\begin{itemize}
	\item Lenguaje Multiparadigma
	\item Concurrencia, hilos de pesos ultraligeros
	\item Lenguaje flexible
\end{itemize}

\section{Desventajas}
\begin{itemize}
	\item Debido a la flexibilidad es un poco lento, se han realizado pruebas donde OZ es 50\% m\'as lento que un compilador de C
\end{itemize}

\begin{thebibliography}{1}
	
	\bibitem{OzHomePage}
	Oz Mozart Home Page, http://mozart.github.io/
	
	\bibitem{MozartProgrammingSystem}
	\emph{Mozart Programming System}, P. Alarcon, H. Spakes, J. Ward; Arkansas Tech University
	
	\bibitem{TutorialOz}
	\emph{Tutorial of Oz}, S. Haridi, N. Franzén, Recuperado de:
	http://mozart.github.io/mozart-v1/doc-1.4.0/tutorial/index.html
\end{thebibliography}

\appendices
\onecolumn
\section{}
\begin{figure}
	\centering
	\includegraphics[width=\textwidth]{datos.png}
	\caption{\label{fig:Datos}Tipos de Datos}
\end{figure}

\end{document}


