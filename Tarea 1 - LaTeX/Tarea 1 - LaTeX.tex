\documentclass[10pt,journal,compsoc]{IEEEtran}
\usepackage[spanish]{babel}
\usepackage{enumitem}
\usepackage{enumerate}
\usepackage{lipsum}
\usepackage{graphicx}
\usepackage{background}
\usepackage{xcolor}
\usepackage{listings}
\lstset
{
	language=[LaTeX]TeX,
	breaklines=true,
	basicstyle=\tt\scriptsize,
	keywordstyle=\color{blue},
	identifierstyle=\color{magenta},
}

\ifCLASSOPTIONcompsoc
  \usepackage[nocompress]{cite}
\else
  \usepackage{cite}
\fi

\ifCLASSINFOpdf
\else
\fi
\newcommand\MYhyperrefoptions{bookmarks=true,bookmarksnumbered=true,
pdfpagemode={UseOutlines},plainpages=false,pdfpagelabels=true,
colorlinks=true,linkcolor={black},citecolor={black},urlcolor={black},
pdftitle={\LaTeX},
pdfsubject={Tarea Corta 1, Latex},
pdfauthor={Daniel, Wilbert, Anthony, Bryan},
}

\hyphenation{op-tical net-works semi-conduc-tor}


\backgroundsetup{contents={\includegraphics[scale=0.015, angle=275]{ITCR_LOGO.png}}}

\begin{document}
\title{\LaTeX}

\author{Daniel~Delgado,~\IEEEmembership{Estudiante,~ITCR,}
        Wilbert~Gonzales,~\IEEEmembership{Estudiante,~ITCR,}
        Anthony~Leandro,~\IEEEmembership{Estudiante,~ITCR,}
        and~Bryan~Mena,~\IEEEmembership{Estudiante,~ITCR}
}
\markboth{Lenguajes de Programaci\'on, Tarea Corta 1, Agosto 2017}
{Shell \MakeLowercase{\textit{et al.}}: \LaTex}
\IEEEtitleabstractindextext{
\begin{abstract}
En este documento, se encuentra recopilada alguna informaci\'on \'util con respecto al uso de LaTeX
\end{abstract}
}

\maketitle

\IEEEdisplaynontitleabstractindextext

\IEEEpeerreviewmaketitle

\section{Datos his\'oricos sobre LaTeX}

\section{Importancia y usos acad\'emicos}
\IEEEPARstart{L}{aTeX} permite concentrarse en lo que es verdaderamente importante: el contenido. Como escritor (cient\'ifico, investigador,  estudiante o no) esta herramienta permite minimizar el tiempo dedicado al diseño del documento y enfocarse en las palabras. Una ventaja considerable, sobre otros sistemas tradicionales, es la alta calidad tipogr\'afica de los documentos que se podr\'an producir.

Con \LaTeX\ es posible escribir art\'iculos para revistas, reportes t\'ecnicos, libros e incluso presentaciones. Puede conseguirse la edici\'on de grandes documentos de manera sencilla, empleando la opci\'on de secciones a lo largo del texto. Una de las funciones m\'as \'utiles es la tipograf\'ia para f\'ormulas matem\'aticas complejas. Existe la posibilidad de generar la bibliograf\'ia autom\'aticamente, lo cual facilita mucho este trabajo. Acad\'emicamente, la opci\'on de utilizar m\'as de un idioma en la edici\'on del texto es muy importante, considerando que los trabajos realizados en \LaTeX\ ser\'an creados por personas de m\'ultiples nacionalidades.

\section{Estilos Importantes}
\begin{itemize}
	\item IEEE
	\begin{itemize}
		\item IEEE define gran variedad de estilos dependiendo del tipo de trabajo a realizar.
		\item Algunos de los estilos m\'as utilizados:
		\begin{itemize}
			\item Transactions 
			\begin{itemize}
				\item Empleado para art\'iculos breves, cortos o sobre alguna comunicaci\'on.
			\end{itemize}
			\item Computer Society
			\begin{itemize}
				\item Formato altamente utilizado para art\'iculos que se van a presentar a revisi\'on.
			\end{itemize}
		\end{itemize}
	\end{itemize}
	\item Memoir
	\begin{itemize}
		\item Memoir fue publicado en el 2001 y actualmente se encuentra en la tercera edici\'on.
		\item Un aspecto a resaltar es la posibilidad de reemplazar otros estilos como book y report obteniendo resultados muy similares.
		\item Hay diferentes sub-estilos con los cuales pueden crearse diseños apropiados para gran cantidad de proyectos, como art\'iculos, tesis, etc.
	\end{itemize}
	\item Beamer
	\begin{itemize}
		\item Esta es una plantilla para crear presentaciones con un m\'inimo esfuerzo.
		\item Beamer cuenta con diferentes ejemplos para muchas de las funciones com\'unmente usadas en presentaciones: t\'itulo, teoremas, figuras, citas, referencias, etc.
		\item Incluso se incluye la opci\'on de utilizar temas y colores.
	\end{itemize}
\end{itemize}

\renewcommand{\thesubsection}{\thesection.\alph{subsection}}

\section{C\'omo crear}
	\subsection{P\'arrafos}	
	\subsection{Efectores de letras, colores}
	\subsection{Tildes, caracteres especiales de distintos idiomas}
	\subsection{T\'itulos, subt\'itulos}
	\subsection{Referencias}
	\subsection{Marcas de agua}
	Hay diferentes paquetes para insertar marcas de agua. Estas pueden ser detr\'as del texto, o delante del texto. El ejemplo del documento aplica una imagen detr\'as del texto, m\'as espec\'ificamente, asignando la marca al \emph{background}. A continuaci\'on se explica c\'omo se puede lograr esto:
	\begin{itemize}
		\item Im\'agenes
		\begin{itemize}
			\item Para utilizar im\'agenes como marcas de agua en el documento deben utilizarse los paquetes \textbf{\textit{graphicx}} y \textbf{\textit{background}}.	
		\end{itemize}
	\begin{lstlisting}
\usepackage{graphicx}
\usepackage{background}
\backgroundsetup{contents={\includegraphics[scale=X]{<archivo>}}}
	\end{lstlisting}
	\item Texto
		\begin{itemize}
			\item Para utilizar texto como marca de agua son necesarios los paquetes \textbf{\textit{draftwatermark}}, para posicionarlo detr\'as del texto, o \textbf{\textit{xwatermark}} para situarlo frente al texto.
			\item Una gu\'ia m\'as detallada y extensa puede encontrarse en \emph{http://ctan.math.washington.edu/tex-archive/macros/latex/contrib/xwatermark/doc/xwatermark-guide.pdf}
			\item El siguiente c\'odigo de ejemplo utiliza \textbf{\textit{xwatermark}}.
			\begin{itemize}
				\item allpages : todas las p\'aginas tendr\'an la marca de agua.
				\item color : selecciona el color y transparencia del texto.
				\item angle : el \'angulo de la marca de agua.
				\item scale : la escala con base en la imagen original.
				\item xpos/ypos : posici\'on en el eje x/y.
				\item Entre corchetes se ingresa el texto deseado.
			\end{itemize}	
			\begin{lstlisting}
\usepackage[printwatermark]{xwatermark}
\newwatermark[allpages,color=red!50,angle=45,scale=3,xpos=0,ypos=0]{TEXTO_DESEADO}
			\end{lstlisting}
		\end{itemize}
	\end{itemize}

\section{Cuadros o tablas}
\IEEEPARstart{L}as tablas son elementos comunes en la mayor\'ia de documentos cient\'ificos, \LaTeX\ provee una gran cantidad de herramientas para personalizar las tablas, modificar el tama\'no o el color de las celdas, entre otros. 

\begin{center}
	\begin{tabular}{||c c c c||} 
		\hline
		Col1 & Col2 & Col3 & Col4 \\ [0.5ex] 
		\hline
		1 & 6 & 87837 & 787 \\ 
		\hline
		2 & 7 & 78 & 5415 \\
		\hline
		3 & 545 & 778 & 7507 \\
		\hline
		4 & 545 & 18744 & 7560 \\
		\hline
		5 & 88 & 788 & 6344 \\ [1ex] 
		\hline
	\end{tabular}
\end{center}

Para crear una tabla como la anterior se hace de la siguiente manera:
	\begin{lstlisting}
\begin{center}
	\begin{tabular}{||c c c c||} 
		\hline
		Col1 & Col2 & Col3 & Col4 \\ [0.5ex] 
		\hline
		1 & 6 & 87837 & 787 \\ 
		\hline
		2 & 7 & 78 & 5415 \\
		\hline
		3 & 545 & 778 & 7507 \\
		\hline
		4 & 545 & 18744 & 7560 \\
		\hline
		5 & 88 & 788 & 6344 \\ [1ex] 
		\hline
	\end{tabular}
\end{center}
	\end{lstlisting}
En el c\'odigo previo, \emph{hline} se refiere a cada fila de la tabla. Luego de definir la fila, cada elemento de la respectiva columna se encuentra dividida por un \emph{\&}. Es posible combinar filas y columnas, crear tablas en m\'ultiples p\'aginas, entre muchas otras opciones de personalizaci\'on.

\section{Ecuaciones Matem\'aticas}
\IEEEPARstart{E}l uso de ecuaciones matem\'aticas en \LaTeX\ es uno de los fuertes de esta herramienta. Hay dos maneras distintas para escribir ecuaciones: \emph{inline} y \emph{display}. La primera es para ecuaciones que son parte del texto y el segundo para los que no son parte del texto o el p\'arrafo, por lo tanto son colocados en l\'ineas separadas.

Podemos insertar la ecuaci\'on de equivalencia de Albert Einsten, $E=mc^2$, en la misma l\'inea o separado del p\'arrafo sin numerar
$$E=mc^2$$
o tambi\'en puede agrerarse la numeraci\'on a una ecuaci\'on
\begin{equation}
	E=mc^2
\end{equation}
El c\'odigo para hacer esto es el siguiente:
\begin{itemize}
	\item Inline
\begin{lstlisting}
...Albert Einsten, $E=mc^2$, en la misma...
\end{lstlisting}
	\item Display sin numerar
\begin{lstlisting}
$$E=mc^2$$
\end{lstlisting}
	\item Display numerado
\begin{lstlisting}
\begin{equation}
	E=mc^2
\end{equation}
\end{lstlisting}
\end{itemize}

\end{document}


