\documentclass[10pt,journal,compsoc]{IEEEtran}
\usepackage[spanish]{babel}
\usepackage[usenames, dvipsnames]{color}
\usepackage{listings}
\usepackage{graphicx}
\usepackage{wrapfig}
\usepackage{xcolor}
\usepackage{enumitem}
\usepackage{enumerate}
\usepackage{lipsum}
\usepackage{graphicx}
\usepackage{background}
\usepackage{color}
\ifCLASSOPTIONcompsoc
  \usepackage[nocompress]{cite}
\else
  \usepackage{cite}
\fi

\lstset
{
	language=[LaTeX]TeX,
	breaklines=true,
	basicstyle=\tt\scriptsize,
	keywordstyle=\color{blue},
	identifierstyle=\color{magenta},
}

\ifCLASSINFOpdf
\else
\fi
\newcommand\MYhyperrefoptions{bookmarks=true,bookmarksnumbered=true,
pdfpagemode={UseOutlines},plainpages=false,pdfpagelabels=true,
colorlinks=true,linkcolor={black},citecolor={black},urlcolor={black},
pdftitle={\LaTeX},
pdfsubject={Tarea Corta 1, Latex},
pdfauthor={Daniel, Wilbert, Anthony, Bryan},
}

\hyphenation{op-tical net-works semi-conduc-tor}

\backgroundsetup{contents={\includegraphics[scale=0.015, angle=275]{ITCR_LOGO.png}}}

\begin{document}
\title{\LaTeX}

\author{Daniel~Delgado,~\IEEEmembership{Estudiante,~ITCR,}
        Wilbert~Gonzales,~\IEEEmembership{Estudiante,~ITCR,}
        Anthony~Leandro,~\IEEEmembership{Estudiante,~ITCR,}
        and~Bryan~Mena,~\IEEEmembership{Estudiante,~ITCR}
}
\markboth{Lenguajes de Programaci\'on, Tarea Corta 1, Agosto 2017}
{Shell \MakeLowercase{\textit{et al.}}: \LaTex}
\IEEEtitleabstractindextext{
\begin{abstract}
En este documento, se encuentra recopilada alguna informaci\'on \'util con respecto al uso de LaTeX
\end{abstract}
}

\maketitle

\IEEEdisplaynontitleabstractindextext

\IEEEpeerreviewmaketitle

\section{Datos hist\'oricos sobre LaTeX}
\IEEEPARstart{T}{ex} es un lenguaje de programaci\'on creado para ser usado en la composici\'on tipogr\'afica. En especial para la composici\'on tipogr\'afica de las matem\'aticas, su creador fue Donald Knuth, quien escribi\'o Tex en 1978.
En 1979 se realiz\'o la conferencia “Josiah Willard Gibbs Lecture” y Tex se hizo popular despu\'es de este evento.
En 1986 Leslie Lamport escribe LaTeX que es un sistema de preparaci\'on de documentos.

A principios de los 90 se impartieron varios cursos de TeX y LaTex. Ademas se utilizaron plantillas con el fin de generar Slides con texto matem\'atico de manera similar a Beamer.
Se imparti\'o un taller con el fin de generar ex\'amenes de juguete con TeX y Pascal.
Editores:
\begin{itemize}
	\item Texshell
	\item Winedt
	\item TeXnicCenter
	\item Emacs
	\item Texmaker
	\item TexStudio
	\item TexLive
\end{itemize}
Distribuci\'on:
\begin{itemize}
	\item MikTeX y fpTeX para Windows
	\item emTeX para DOS/OS2
	\item OzTeX para Mac
\end{itemize}

\section{Importancia y usos acad\'emicos}
\IEEEPARstart{L}{aTeX} permite concentrarse en lo que es verdaderamente importante: el contenido. Como escritor (cient\'ifico, investigador,  estudiante o no) esta herramienta permite minimizar el tiempo dedicado al diseño del documento y enfocarse en las palabras. Una ventaja considerable, sobre otros sistemas tradicionales, es la alta calidad tipogr\'afica de los documentos que se podr\'an producir.

Con \LaTeX\ es posible escribir art\'iculos para revistas, reportes t\'ecnicos, libros e incluso presentaciones. Puede conseguirse la edici\'on de grandes documentos de manera sencilla, empleando la opci\'on de secciones a lo largo del texto. Una de las funciones m\'as \'utiles es la tipograf\'ia para f\'ormulas matem\'aticas complejas. Existe la posibilidad de generar la bibliograf\'ia autom\'aticamente, lo cual facilita mucho este trabajo. Acad\'emicamente, la opci\'on de utilizar m\'as de un idioma en la edici\'on del texto es muy importante, considerando que los trabajos realizados en \LaTeX\ ser\'an creados por personas de m\'ultiples nacionalidades.

\section{Estilos Importantes}
\begin{itemize}
	\item IEEE
	\begin{itemize}
		\item IEEE define gran variedad de estilos dependiendo del tipo de trabajo a realizar.
		\item Algunos de los estilos m\'as utilizados:
		\begin{itemize}
			\item Transactions 
			\begin{itemize}
				\item Empleado para art\'iculos breves, cortos o sobre alguna comunicaci\'on.
			\end{itemize}
			\item Computer Society
			\begin{itemize}
				\item Formato altamente utilizado para art\'iculos que se van a presentar a revisi\'on.
			\end{itemize}
		\end{itemize}
	\end{itemize}
	\item Memoir
	\begin{itemize}
		\item Memoir fue publicado en el 2001 y actualmente se encuentra en la tercera edici\'on.
		\item Un aspecto a resaltar es la posibilidad de reemplazar otros estilos como book y report obteniendo resultados muy similares.
		\item Hay diferentes sub-estilos con los cuales pueden crearse diseños apropiados para gran cantidad de proyectos, como art\'iculos, tesis, etc.
	\end{itemize}
	\item Beamer
	\begin{itemize}
		\item Esta es una plantilla para crear presentaciones con un m\'inimo esfuerzo.
		\item Beamer cuenta con diferentes ejemplos para muchas de las funciones com\'unmente usadas en presentaciones: t\'itulo, teoremas, figuras, citas, referencias, etc.
		\item Incluso se incluye la opci\'on de utilizar temas y colores.
	\end{itemize}
\end{itemize}

\section{Como Crear...}


\subsection{P\'arrafos}
Desde el punto de vista de las personas, un p\'arrafo es la uni\'on de varias palabras las cuales se ubican despu\'es de un t\'itulo o punto y aparte, hasta el siguiente punto y aparte. En \LaTeX\ se pueden hacer p\'arrafos simplemente escribiendo despu\'es de una secci\'on o subsecci\'on.\\  Adem\'as, \LaTeX\ maneja p\'arrafos bajo una estructura propia. Para comenzar a escribir un p\'arrafo, es necesario escribir la etiqueta correspondiente para p\'arrafos, la cu\'al es la siguiente: 
\begin{lstlisting}
\paragraph{Titulo} 
\end{lstlisting}
Un ejemplo de un p\'arrafo implementado bajo esta etiqueta es el siguiente:
\paragraph{Este es el t\'itulo} 
Los estudiantes de segundo a\~no del T\'ecnol\'ogico de Costa Rica que cursan el segundo semestre de la carrera de Ingenier\'ia en Computaci\'on deber\'ian estar cursando en este momento el curso IC-4700 el cu\'al corresponde a Lenguajes de Programaci\'on.

\subsection{Efectos de letras, colores}
El estilo de letra se puede ajustar a las necesidades de los usuarios en diferentes formas: \\
\begin{itemize}
	
	\item \'Enfasis: Se usa por medio del comando \\
	\begin{lstlisting}
	\emph{texto}
	\end{lstlisting}
	Adem\'as, este comando puede combinarse consigo mismo para devolver texto a la forma normal (sin enfatizar) entre un gran texto que s\'i tiene todo un \'enfasis de inicio a fin. Por ejemplo: \\
	\emph{Dentro de esta frase existe una \emph{palabra} que no ha sido enfatizada.}
	
	\item It\'alico: Se usa por medio del comando \\
	\begin{lstlisting}
	\textit{texto}
	\end{lstlisting}
	Un ejemplo del uso de letra con efecto it\'alico es el siguiente: \\
	Una serie que tiene muchos seguidores a nivel mundial es \textit{Game of Thrones}.
	
	\item Negrita: Se usa por medio del comando: \\
	\begin{lstlisting}
	\textbf{texto}
	\end{lstlisting}
	Un ejemplo del uso de la negrita en un texto es el siguiente: \\
	Las personas que se \textbf{esfuerzan} y \textbf{luchan} por sus sue\~nos consiguen una gran recompensa.
	
	\item Letras min\'usculas: Sirve para poner letras en may\'uscula y con un tama\~no m\'as peque\~no de lo tradicional. Se usa por medio del comando \\
	\begin{lstlisting}
	\textsc{texto}
	\end{lstlisting}
	Un ejemplo del uso de este efecto es el siguiente: \\
	No se lee igual la palabra \textsc{tec}, que TEC. 
	
	\item Inclinado: Se usa por medio del comando: \\
	\begin{lstlisting}
	\textsl{texto}
	\end{lstlisting}
	Por ejemplo: \textsl{Este texto est\'a inclinado.}
	\item Anidado: Significa combinar un estilo con otro en el mismo texto. Por ejemplo: \\
	\begin{lstlisting}
	\emph{\textbf{texto}}
	\end{lstlisting}
	Por ejemplo: el comando de arriba \emph{\textbf{quedar\'ia as\'i}}.
	
	\item Colores: Para utilizar colores en las letras, primero hay que importar el paquete \textit{\textbf{color}}. Esto se hace por medio del comando: \\
	\begin{lstlisting}
	\usepackage{color}
	\end{lstlisting}
	Es preferible utilizar este comando antes del inicio del documento.
	Seguidamente, se necesita utilizar el siguiente comando: \\
	\begin{lstlisting}
	\textcolor{color}{texto}
	\end{lstlisting}
	Y as\'i, ya se puede utilizar \textcolor{red}{colores} en el texto.
	
\end{itemize}

\subsection{Efectos de letras y colores}
Para cambiar texto de color se utiliza el paquete color \verb|\usepackage[usenames, dvipsnames]{color}| se utiliza l comando \verb|\color{Color}| para cambiar el color del texto desde el punto en que se utiliza, adem\'as se puede utlizar el comando \verb|\pagecolor{Color}| para cambiar el color de la p\'agina.\\
Para cambiar colores de fondo y texto tambien se utilizan los comandos \verb|\textcolor{Color}{Texto}| para cambiar el color del texto, \verb|\colorbox{color}{texto}| para ponerle fondo de color al texto, \verb|\fcolorbox{ColorBorde}{ColorFondo}{texto}| para encerrar un texto en una caja con color de borde y de fondo:\\
\textcolor{red}{Texto}\\\newline
\colorbox{blue}{Texto}\\\newline
\fcolorbox{blue}{green}{Texto}\\\newline

Para cambiar el estilo del texto se utilizan los comandos \verb|\textbf{Texto}| para texto en negrita, \verb|\textit{Texto}| para texto en it\'alica, \verb|\textsl{Texto}| para texto inclinado y \verb|\texttt{Texto}| para texto en estilo. Ejemplos:\\
Negrita: \textbf{Texto}\\
Italica: \textit{Texto}\\
Inclinado: \textsl{Texto}\\
Estilo: \texttt{Texto}\\
Para tama\~no de letra se utilizan los comandos
\begin{lstlisting}
\tiny
\scriptsize
\small
\normalsize
\large
\Large
\LARGE
\huge
\Huge
\end{lstlisting}
Ejemplos:\\
\verb|\tiny|: \tiny texto\\\normalsize
\verb|\scriptsize|: \scriptsize texto\\\normalsize
\verb|\small|: \small texto\\\normalsize
\verb|\normalsize|: \normalsize texto\\\normalsize
\verb|\large|: \large texto\\\normalsize
\verb|\Large|: \Large texto\\\normalsize
\verb|\LARGE|: \LARGE texto\\\normalsize
\verb|\huge|: \huge texto\\\normalsize
\verb|\Huge|: \Huge texto\\\normalsize
\normalsize

\subsection{Tildes y caracteres especiales}
Para utilizar otros caracteres especiales de idiomas diferentes como chino o ruso existe un paquete llamado Babel creado por Johannes Braams y Javier Bezos, para utilizar babel se utiliza el siguiente comando: \verb|\usepackage[language]{babel}| es recomendado utilizar este comando inmediatamente despues de utilizar \verb|\documentclass| asi otros paquetes sabran que lenguaje se esta utilizando. Babel permite ser "invocado"utilizando varios lenguajes como parametros, de esta forma si tenemos el comando \verb|\usepackage[languageA,languageB]{babel}| se tomará el "languageB" como el lenguaje activo y se podrá cambiar uttilizando \verb|\selectlanguage{languageA}|, además, babel permite utilizar ciertos comandos como \verb|\foreignlanguage{language}{Texto}| y \verb|\begin{otherlanguage}...\end{otherlanguage}| para encapsular texto escrito en otros idiomas.\newline\newline
\begin{minipage}{0.2\textwidth}
	\begin{tabular}{ l r r }
		\LaTeX & Resultado \\
		\hline\\
		\verb|\'e| & \'e \\
		\verb|\=e| & \=e \\
		\verb|\u e|& \u e\\
		\verb|\d e|& \d e \\
		\verb|\oe| & \oe \\
		\verb|\aa| & \aa \\
		\verb|\i| & \i \\
		\verb|\l| & \l \\
		\verb|\`e| & \`e \\
		\verb|\.e| & \.e \\
		\verb|\v e| & \v e \\
		\verb|\b e| & \b e \\
		\verb|\OE| & \OE \\
		\verb|\AA| & \AA \\
	\end{tabular}
	\hfill
\end{minipage}
\begin{minipage}{0.1\textwidth}
	\begin{tabular}{ l r r }
		\LaTeX & Resultado \\
		\hline\\
		\verb|\j| & \j \\
		\verb|\L| & \L \\
		\verb|\l| & \l \\
		\verb|\^e| & \^e \\
		\verb|\"e| & \"e \\
		\verb|\c e| & \c e \\
		\verb|\t ee|& \t ee \\
		\verb|\ae| & \ae \\
		\verb|\o| & \o \\
		\verb|\ss| & \ss \\
		\verb|\~e| & \~e \\
		\verb|\H e| & \H e \\
		\verb|\r e| & \r e \\
		\verb|\AE| & \AE \\
		\verb|\O|& \O \\
	\end{tabular}
	\hfill
\end{minipage}\\
Para una lista m\'as detallada (Simbolos matem\'aticos, griegos, entre otros) revisar \emph{Latex A Document Preparation System User's Guide and Reference Manual}\textsuperscript{[2]} p\'ag. 41


\subsection{T\'itulos y subt\'itulos}
El t\'itulo se agrega al inicio del documento, justo despu\'es del comando: \\
\begin{lstlisting}
\begin{document}
\end{lstlisting}
Para agregar un t\'itulo, se debe utilizar el comando: \\
\begin{lstlisting}
\title{Titulo}
\end{lstlisting}
Y para agregar un subt\'itulo luego del t\'itulo, se utiliza: \\
\begin{lstlisting}
\subtitle{Subtitulo}
\end{lstlisting}

\subsection{Marcas de agua}
Hay diferentes paquetes para insertar marcas de agua. Estas pueden ser detr\'as del texto, o delante del texto. El ejemplo del documento aplica una imagen detr\'as del texto, m\'as espec\'ificamente, asignando la marca al \emph{background}. A continuaci\'on se explica c\'omo se puede lograr esto:
\begin{itemize}
	\item Im\'agenes
	\begin{itemize}
		\item Para utilizar im\'agenes como marcas de agua en el documento deben utilizarse los paquetes \textbf{\textit{graphicx}} y \textbf{\textit{background}}.	
	\end{itemize}
	\begin{lstlisting}
	\usepackage{graphicx}
	\usepackage{background}
	\backgroundsetup{contents={\includegraphics[scale=X]{<archivo>}}}
	\end{lstlisting}
	\item Texto
	\begin{itemize}
		\item Para utilizar texto como marca de agua son necesarios los paquetes \textbf{\textit{draftwatermark}}, para posicionarlo detr\'as del texto, o \textbf{\textit{xwatermark}} para situarlo frente al texto.
		\item Una gu\'ia m\'as detallada y extensa puede encontrarse en \emph{http://ctan.math.washington.edu/tex-archive/macros/latex/contrib/xwatermark/doc/xwatermark-guide.pdf}
		\item El siguiente c\'odigo de ejemplo utiliza \textbf{\textit{xwatermark}}.
		\begin{itemize}
			\item allpages : todas las p\'aginas tendr\'an la marca de agua.
			\item color : selecciona el color y transparencia del texto.
			\item angle : el \'angulo de la marca de agua.
			\item scale : la escala con base en la imagen original.
			\item xpos/ypos : posici\'on en el eje x/y.
			\item Entre corchetes se ingresa el texto deseado.
		\end{itemize}	
		\begin{lstlisting}
		\usepackage[printwatermark]{xwatermark}
		\newwatermark[allpages,color=red!50,angle=45,scale=3,xpos=0,ypos=0]{TEXTO_DESEADO}
		\end{lstlisting}
	\end{itemize}
\end{itemize}

\subsection{Headers.}
Los Headers  para un documento IEEEtran en {\LaTeX} se realizan por medio del siguiente c\'odigo.
\begin{lstlisting}
\markboth{Journal of \LaTeX\ Class Files,~Vol.~14, No.~8, August~2015}%
{Shell \MakeLowercase{\textit{et al.}}: Bare Advanced Demo of IEEEtran.cls for IEEE Computer Society Journals}
\end{lstlisting}

\subsection{Footers.}
Los agradecimientos se agregan al pie de p\'agina en {\LaTeX} se realizan por medio del siguiente c\'odigo.
\begin{lstlisting}
\IEEEcompsocitemizethanks{\IEEEcompsocthanksitem M. Shell was with the Department
of Electrical and Computer Engineering, Georgia Institute of Technology, Atlanta,
GA, 30332.\protect\\
E-mail: see http://www.michaelshell.org/contact.html
\IEEEcompsocthanksitem J. Doe and J. Doe are with Anonymous University.}
\thanks{Manuscript received April 19, 2005; revised August 26, 2015.}}
\end{lstlisting}

\subsection{Saltos de p\'agina, columnas y citas al pie de p\'agina.}
Los Saltos de p\'agina  en {\LaTeX} se realizan por medio del siguiente c\'odigo.

\begin{lstlisting}
\newpage
\end{lstlisting}

Las columnas  en {\LaTeX} se realizan por medio del siguiente paquete.
\begin{lstlisting}
\usepackage{multicol}
\end{lstlisting}
Para hacer que el texto se muestre en columnas, se debe de introducir dentro de los 2 c\'odigos de LaTeX. El n\'umero indica la cantidad de columnas que se necesita.
\begin{lstlisting}
\begin{multicols}{2}

TEXTO COLUMNA

\end{multicols}
\end{lstlisting}
Ademas podemos mostrar columnas dentro de columnas.
\begin{lstlisting}
\begin{multicols}{2}

TEXTO COLUMNA

\begin{multicols}{2}

TEXTO INTRACOLUMNA

\end{multicols}

TEXTO COLUMNA

\end{multicols}
\end{lstlisting}
Para dejar un espacio detr\'as y delante de las columnas
\begin{lstlisting}
\setlength{multicolsep}{3cm}begin{multicols}{2}

TEXTO COLUMNA

\end{multicols}

** Es importante reestablecer el valor predeterminado.

\setlength{multicolsep}{13pt}
\end{lstlisting}
Si se requiere una linea de separaci\'on entre columnas.
\begin{lstlisting}
\setlength{columnseprule}{2pt}begin{multicols}{2}

TEXTO COLUMNA

\end{multicols}

** Es importante reestablecer el valor predeterminado.

\setlength{columnseprule}{0pt}
\end{lstlisting}
Tambi\'en se puede hacer la separaci\'on de columnas sin la linea.
\begin{lstlisting}
\setlength{columnsep}{3cm}begin{multicols}{2}

TEXTO COLUMNA

\end{multicols}

** Es importante reestablecer el valor predeterminado.

\setlength{columnsep}{10pt}
\end{lstlisting}

En vez de situar cada nota al pie de la p\'agina, se prefiere agrupar todas al final del documento, para lograr esto debemos de usar el paquete
\begin{lstlisting}
\usepackage{endnotes}
\end{lstlisting}
Con este paquete vamos indicando cada nota con
\begin{lstlisting}
\endnote{}
\end{lstlisting}
Y donde se quieran mostrar, se escribe
\begin{lstlisting}
\theendnotes
\end{lstlisting}
Por ejemplo:
\begin{lstlisting}
\usepackage{endnotes}

\begin{document}

\chapter{Introducción}
Algunos de los animales en peligro de extinción son el oso blanco\endnote{en el Ártico}, el cóndor\endnote{en los Andes}, el tigre siberiano\endnote{en Siberia}, y el lince ibérico\endnote{en la Península Ibérica}.

\chapter{Final}
\theendnotes

\end{document}
\end{lstlisting}


\section{Cuadros o tablas}
\IEEEPARstart{L}as tablas son elementos comunes en la mayor\'ia de documentos cient\'ificos, \LaTeX\ provee una gran cantidad de herramientas para personalizar las tablas, modificar el tama\'no o el color de las celdas, entre otros. 

\begin{center}
	\begin{tabular}{||c c c c||} 
		\hline
		Col1 & Col2 & Col3 & Col4 \\ [0.5ex] 
		\hline
		1 & 6 & 87837 & 787 \\ 
		\hline
		2 & 7 & 78 & 5415 \\
		\hline
		3 & 545 & 778 & 7507 \\
		\hline
		4 & 545 & 18744 & 7560 \\
		\hline
		5 & 88 & 788 & 6344 \\ [1ex] 
		\hline
	\end{tabular}
\end{center}

Para crear una tabla como la anterior se hace de la siguiente manera:
\begin{lstlisting}
\begin{center}
\begin{tabular}{||c c c c||} 
\hline
Col1 & Col2 & Col3 & Col4 \\ [0.5ex] 
\hline
1 & 6 & 87837 & 787 \\ 
\hline
2 & 7 & 78 & 5415 \\
\hline
3 & 545 & 778 & 7507 \\
\hline
4 & 545 & 18744 & 7560 \\
\hline
5 & 88 & 788 & 6344 \\ [1ex] 
\hline
\end{tabular}
\end{center}
\end{lstlisting}
En el c\'odigo previo, \emph{hline} se refiere a cada fila de la tabla. Luego de definir la fila, cada elemento de la respectiva columna se encuentra dividida por un \emph{\&}. Es posible combinar filas y columnas, crear tablas en m\'ultiples p\'aginas, entre muchas otras opciones de personalizaci\'on.

\section{Como crear figuras y gr\'aficas}
\begin{wrapfigure}{R}{0.2\textwidth}
	\centering
	\includegraphics[width=0.1\textwidth]{pic1.jpg}
	\caption{Figura envuelta en texto}
\end{wrapfigure}
Para insertar gr\'aficos y figuras se utiliza el paquete \verb|\usepackage{graphicx}|. Para insertar figuras se utiliza el siguiente comando
\begin{lstlisting}
\begin{figure}
\centering
\includegraphics{grafico}
\caption{Caption}
\label{Label}
\end{figure}
\end{lstlisting}
donde gr\'afico es el path de la figura que se quiere insertar y caption y label son textos que se le puede insertar a la figura.
Para mayor comodidad y para mejorar la apariencia existe la posibilidad de hacer que el texto envuelva la imagen con el uso de \verb|\usepackage{wrapfig}| y el comando de inicio:\footnote{Ver Figura 1.}
\begin{lstlisting}
\begin{wrapfigure}{i}{0.5\textwidth}
contenido
\end{wrapfigure}
\end{lstlisting}

\section{Minipage}%%%%%%%FIME minipage es seccion #7
Minipage es utilizado para colocar objetos uno al lado de otros, los cuales, sin minipage, ser\'ia muy dificil.
Comando para utilizar minipage:
\begin{lstlisting}
\begin{minipage}[adjusting]{width of the minipage}
Texto ... \ \
Imagenes ... \ \
Tablas ... \ \
\end{minipage}
\end{lstlisting}
Donde adjusting puede tomar valores de c (centrado), t(top), b(bottom)
\subsection{Ejemplos}
\begin{minipage}{0.2\textwidth}
	\begin{tabular}{|c|c|c|}
		\hline
		A & B & C \\
		\hline
		1 & 2 & 3  \\
		\hline 
		4 & 5 & 6 \\
		\hline
	\end{tabular}
\end{minipage}
\begin{minipage}{0.2\textwidth}
	\begin{tabular}{c|c|c}
		A & B & C \\
		\hline
		1 & 2 & 3  \\
		\hline 
		4 & 5 & 6 \\
	\end{tabular}
\end{minipage}\\
\newline
Dos tablas juntas\\
\begin{minipage}{0.2\textwidth}
	\begin{tabular}{c|c|c}
		A & B & C \\
		\hline
		1 & 2 & 3  \\
		\hline 
		4 & 5 & 6 \\
	\end{tabular}
\end{minipage}
\begin{minipage}{0.2\textwidth}
	\includegraphics[width=\textwidth]{pic1.jpg}
\end{minipage}\\
\newline
Tabla junto a imagen\\

\section{C\'omo crear Bibliograf\'ias}
Las bibliograf\'ias en {\LaTeX} se realizan por medio del siguiente c\'odigo.

\begin{lstlisting}
\begin{thebibliography}{1}
\bibitem{IEEEhowto:kopka}
H.~Kopka and P.~W. Daly, \emph{A Guide to {\LaTeX}}, 3rd~ed.\hskip 1em plus
0.5em minus 0.4em\relax Harlow, England: Addison-Wesley, 1999.
\end{thebibliography}
\end{lstlisting}

\section{Como crear citas}
Se debe de insertar el paquete
\begin{lstlisting}
\ usepackage {cite}
\end{lstlisting} 
Al inicio del documento de {\LaTeX} para mejorar el uso de citas num\'ericas y se tienen las siguientes opciones:
\subsection{Spacing.}
Se escribe un pequeño espacio despu\'es de comas en la lista de citas.
La opci\'on [nospace] elimina ese espacio y la opci\'on de [space] la cambia con un espacio entre palabras com\'un.
\subsection{Sorting.}
Las citas de una lista son clasificadas de manera ascendente. Con la opci\'on de [nosort] se desactiva la ordenaci\'on. Las citas clasificables tienen que ser num\'ericas o al menos la mayor\'ia num\'ericas. Las entradas que no son clasificables se imprimen antes.
\subsection{Compression.}
Grupos de tres o mas n\'umeros consecutivos son comprimidos en un rango usando un guión. Ejemplo: Se tiene la lista [7,5,6,?,4,9,8,Einstein,6] que se mostrar\'ia como [?,Einstein,4–6,6–9]. La opci\'on de compresi\'on de este paquete se puede deshabilitar con [nocompress].
\subsection{Non-numbers.}
Trabajo de compresi\'on y clasificaci\'on con n\'umeros positivos (8,6,7,9 da [6-9]), as\'i como tambi\'en n\'umeros con caracteres sufijo o prefijo ([5a-5c] o [T1-T4]) y tambi\'en n\'umeros duales con un separador ([1,11 - 1,15]). Los n\'umeros duales no se mezclan bien con n\'umeros individuales. Otras formas de entradas se imprimen antes de todas las formas ordenables.

\section{Ecuaciones Matem\'aticas}
\IEEEPARstart{E}l uso de ecuaciones matem\'aticas en \LaTeX\ es uno de los fuertes de esta herramienta. Hay dos maneras distintas para escribir ecuaciones: \emph{inline} y \emph{display}. La primera es para ecuaciones que son parte del texto y el segundo para los que no son parte del texto o el p\'arrafo, por lo tanto son colocados en l\'ineas separadas.

Podemos insertar la ecuaci\'on de equivalencia de Albert Einsten, $E=mc^2$, en la misma l\'inea o separado del p\'arrafo sin numerar
$$E=mc^2$$
o tambi\'en puede agrerarse la numeraci\'on a una ecuaci\'on
\begin{equation}
E=mc^2
\end{equation}
El c\'odigo para hacer esto es el siguiente:
\begin{itemize}
	\item Inline
	\begin{lstlisting}
	...Albert Einsten, $E=mc^2$, en la misma...
	\end{lstlisting}
	\item Display sin numerar
	\begin{lstlisting}
	$$E=mc^2$$
	\end{lstlisting}
	\item Display numerado
	\begin{lstlisting}
	\begin{equation}
	E=mc^2
	\end{equation}
	\end{lstlisting}
\end{itemize}


\section{C\'omo agregar im\'agenes}

Para agregar im\'agenes se necesita importar el paquete \textit{\textbf{graphicx}} \\
\begin{lstlisting}
\usepackage{graphicx}
\end{lstlisting}
Luego se utiliza el comando: \\
\begin{lstlisting}
\includegraphics{archivo_imagen}
\end{lstlisting}
Y si se prefiere, se puede especificar el ancho y largo de la imagen como par\'ametro en el comando de arriba. \\
\begin{lstlisting}
\includegraphics[width=2cm,height=3cm]{...}
\end{lstlisting}
Ser\'ia un ejemplo de una imagen de 2x3 cm. Un ejemplo de c\'omo se ver\'ia una imagen importada ser\'ia la siguiente: \\
\includegraphics[width=3cm, height=3cm]{aa}
\footnote{Imagen de muestra, tomada de http://albertaalpine.ca}

\begin{thebibliography}{2}
	
	\bibitem{Babel}
	J.~Braams y J.~Bezos \emph{Babel}, Junio 2017. Recuperado de http://ftp.ntou.edu.tw/ctan/macros/latex2e/required/babel/base/babel.pdf  
	
	\bibitem{Latex}
	L.~Lamport \emph{Latex A Document Preparation System User's Guide and Reference Manual} Recuperado de:
	http://users.softlab.ntua.gr/~sivann/books/LaTeX\%20-\%20User's\%20Guide\%20and\%20Reference\%20Manual-lamport94.pdf
	
	\bibitem{HeatersFooters}
	~Luis, \emph{Encabezados y pies de página en {\LaTeX}}.\hskip 1em plus
	0.5em minus 0.4em\relax recuperado de: http://minisconlatex.blogspot.com/2013/01/como-editar-los-encabezados-y-pies-de.html, el 16 de agosto 2017.
	
	\bibitem{Footnotes}
	~Luis, \emph{Notas al pie de página (y al final del documento) en {\LaTeX}}.\hskip 1em plus
	0.5em minus 0.4em\relax recuperado de: http://minisconlatex.blogspot.com/2011/04/notas-al-pie-de-pagina.html, el 16 de agosto 2017.
	
	\bibitem{Historia}
	~Unknown, \emph{Breve historia de {\LaTeX}}.\hskip 1em plus
	0.5em minus 0.4em\relax recuperado de: http://nereida.deioc.ull.es/~pcgull/ihiu01/cdrom/latex/contenido/node2.html, el 20 de agosto 2017.
	
	\bibitem{Columnas}
	~Unknown, \emph{Columnas con el paquete multicol en {\LaTeX}}.\hskip 1em plus
	0.5em minus 0.4em\relax recuperado de: https://bioinformatiquillo.wordpress.com/2009/02/16/lyx-latex-columnas-con-el-paquete-multicol/, el 18 de agosto 2017.
	
	\bibitem{Bibliografia}
	~M. Mata, \emph{Bibliografía en {\LaTeX}}.\hskip 1em plus
	0.5em minus 0.4em\relax recuperado de: http://logistica.fime.uanl.mx/miguel/docs/BibTeX.pdf, el 20 de agosto 2017.
	
	\bibitem{CTAN}
	CTAN: Package book. (s. f.). Recuperado 4 de agosto de 2017, a partir de https://www.ctan.org/pkg/book
	
	\bibitem{CTANReport}	
	CTAN: Package report. (s. f.). Recuperado 4 de agosto de 2017, a partir de https://www.ctan.org/pkg/report
		
	\bibitem{MathExpre}
	Mathematical expressions - ShareLaTeX, Online LaTeX Editor. (s. f.). Recuperado 20 de agosto de 2017, a partir de https://www.sharelatex.com/learn/Mathematical\_expressions
		
	\bibitem{Tables}
	Tables - ShareLaTeX, Online LaTeX Editor. (s. f.). Recuperado 19 de agosto de 2017, a partir de https://www.sharelatex.com/learn/Tables
	
	\bibitem{Memoir}	
	Wilson, P. (2016, mayo 16). The Memoir Class for Configurable Typesetting. Recuperado a partir de http://mirrors.ucr.ac.cr/CTAN/macros/latex/contrib/memoir/memman.pdf\
	
	\bibitem{Guide}
	Kottwitz, S. (2011). Latex beginner's guide. Recuperado de  http://ebookcentral.proquest.com
	
		
\end{thebibliography}

\end{document}


