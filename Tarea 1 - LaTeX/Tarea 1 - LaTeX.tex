\documentclass[10pt,journal,compsoc]{IEEEtran}
\usepackage[spanish]{babel}
\ifCLASSOPTIONcompsoc
  \usepackage[nocompress]{cite}
\else
  \usepackage{cite}
\fi

\ifCLASSINFOpdf
\else
\fi
\newcommand\MYhyperrefoptions{bookmarks=true,bookmarksnumbered=true,
pdfpagemode={UseOutlines},plainpages=false,pdfpagelabels=true,
colorlinks=true,linkcolor={black},citecolor={black},urlcolor={black},
pdftitle={\LaTeX},
pdfsubject={Tarea Corta 1, Latex},
pdfauthor={Daniel, Wilbert, Anthony, Bryan},
}

\hyphenation{op-tical net-works semi-conduc-tor}


\begin{document}
\title{\LaTeX}

\author{Daniel~Delgado,~\IEEEmembership{Estudiante,~ITCR,}
        Wilbert~Gonzales,~\IEEEmembership{Estudiante,~ITCR,}
        Anthony~Leandro,~\IEEEmembership{Estudiante,~ITCR,}
        and~Bryan~Mena,~\IEEEmembership{Estudiante,~ITCR}
}
\markboth{Lenguajes de Programaci\'on, Tarea Corta 1, Agosto 2017}
{Shell \MakeLowercase{\textit{et al.}}: \LaTex}
\IEEEtitleabstractindextext{
\begin{abstract}
En este documento, se encuentra recopilada alguna informaci\'on \'util con respecto al uso de LaTeX
\end{abstract}
}

\maketitle

\IEEEdisplaynontitleabstractindextext

\IEEEpeerreviewmaketitle

\section{Importancia y usos acad\'emicos}
\IEEEPARstart{L}{aTeX} permite concentrarse en lo que es verdaderamente importante: el contenido. Como escritor (cient\'ifico, investigador,  estudiante o no) esta herramienta permite minimizar el tiempo dedicado al diseño del documento y enfocarse en las palabras. Una ventaja considerable, sobre otros sistemas tradicionales, es la alta calidad tipogr\'afica de los documentos que se podr\'an producir.

Con \LaTeX\ es posible escribir art\'iculos para revistas, reportes t\'ecnicos, libros e incluso presentaciones. Puede conseguirse la edici\'on de grandes documentos de manera sencilla, empleando la opci\'on de secciones a lo largo del texto. Una de las funciones m\'as \'utiles es la tipograf\'ia para f\'ormulas matem\'aticas complejas. Existe la posibilidad de generar la bibliograf\'ia autom\'aticamente, lo cual facilita mucho este trabajo. Acad\'emicamente, la opci\'on de utilizar m\'as de un idioma en la edici\'on del texto es muy importante, considerando que los trabajos realizados en \LaTeX\ ser\'an creados por personas de m\'ultiples nacionalidades.

\section{Estilos Importantes}
\begin{itemize}
	\item IEEE
	\begin{itemize}
		\item IEEE define gran variedad de estilos dependiendo del tipo de trabajo a realizar.
		\item Algunos de los estilos m\'as utilizados:
		\begin{itemize}
			\item Transactions 
			\begin{itemize}
				\item Empleado para art\'iculos breves, cortos o sobre alguna comunicaci\'on.
			\end{itemize}
			\item Computer Society
			\begin{itemize}
				\item Formato altamente utilizado para art\'iculos que se van a presentar a revisi\'on.
			\end{itemize}
		\end{itemize}
	\end{itemize}
	\item Memoir
	\begin{itemize}
		\item Memoir fue publicado en el 2001 y actualmente se encuentra en la tercera edici\'on.
		\item Un aspecto a resaltar es la posibilidad de reemplazar otros estilos como book y report obteniendo resultados muy similares.
		\item Hay diferentes sub-estilos con los cuales pueden crearse diseños apropiados para gran cantidad de proyectos, como art\'iculos, tesis, etc.
	\end{itemize}
	\item Beamer
	\begin{itemize}
		\item Esta es una plantilla para crear presentaciones con un m\'inimo esfuerzo.
		\item Beamer cuenta con diferentes ejemplos para muchas de las funciones com\'unmente usadas en presentaciones: t\'itulo, teoremas, figuras, citas, referencias, etc.
		\item Incluso se incluye la opci\'on de utilizar temas y colores.
	\end{itemize}
\end{itemize}

\begin{table}
		\centering
		\caption{Simbolos utiles en \LaTeX}
		\begin{tabular}{ l r r }
			Codigo \LaTeX & Resultado \\
			\hline\\
			 \verb|\'e| & \'e \\
			 \verb|\=e| & \=e \\
			 \verb|\u e|& \u e\\
			 \verb|\d e|& \d e \\
			 \verb|oe| & \oe \\
			 \verb|aa| & \aa \\
			 \verb|i| & \i \\
			 \verb|l| & \l \\
			 \verb|\`e| & \`e \\
			 \verb|\.e| & \.e \\
			 \verb|v e| & \v e \\
			 \verb|b e| & \b e \\
			 \verb|OE| & \OE \\
			 \verb|AA| & \AA \\
			 \verb|j| & \j \\
			 \verb|L| & \L \\
			 \verb|l| & \l \\
			 \verb|\^e| & \^e \\
			 \verb|\"e| & \"e \\
			 \verb|c e| & \c e \\
			 \verb|t ee|& \t ee \\
			 \verb|\ae| & \ae \\
			 \verb|o| & \o \\
			 \verb|ss| & \ss \\
			 \verb|~e| & \~e \\
			 \verb|H e| & \H e \\
			 \verb|r e| & \r e \\
			 \verb|AE| & \AE \\
			 \verb|O|& \O \\
		\end{tabular}
	\hfill
\end{table}
Para una lista m\'as detallada (Simbolos matematicos, griegos, entre otros) revisar \emph{Latex A Document Preparation System User's Guide and Reference Manual}[2] pág. 41
Para utilizar otros caracteres especiales de idiomas diferentes como chino o ruso existe un paquete llamado Babel creado por Johannes Braams y Javier Bezos, para utilizar babel se utiliza el siguiente comando: \verb|\usepackage[language]{babel}| es recomendado utilizar este comando inmediatamente despues de utilizar \verb|\documentclass| asi otros paquetes sabran que lenguaje se esta utilizando. Babel permite ser "invocado"utilizando varios lenguajes como parametros, de esta forma si tenemos el comando \verb|\usepackage[languageA,languageB]{babel}| se tomará el "languageB" como el lenguaje activo y se podrá cambiar uttilizando \verb|\selectlanguage{languageA}|, además, babel permite utilizar ciertos comandos como \verb|\foreignlanguage{language}{Texto}| y \verb|\begin{otherlanguage}...\end{otherlanguage}| para encapsular texto escrito en otros idiomas.

\appendices
\section{}
Appendix one text goes here.

\section{}
Appendix two text goes here.

\begin{thebibliography}{2}
	
	\bibitem{Babel}
	J.~Braams y J.~Bezos \emph{Babel}, Junio 2017. Recuperado de http://ftp.ntou.edu.tw/ctan/macros/latex2e/required/babel/base/babel.pdf  
	
	\bibitem{Latex}
	L.~Lamport \emph{Latex A Document Preparation System User's Guide and Reference Manual} Recuperado de:
	http://users.softlab.ntua.gr/~sivann/books/LaTeX\%20-\%20User's\%20Guide\%20and\%20Reference\%20Manual-lamport94.pdf
	
	
	
\end{thebibliography}

\end{document}


